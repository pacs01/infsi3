\section{Web Entry Server}
Ohne Massnahmen werden mehrere Verbindungen direkt zu den Applikationsserver in der DMZ gemacht. Um dies zu verhindern wird eine \textbf{Web Application Firewall}(WAF) eingesetzt. Diese kann folgende Funktionen übernehmen:
\begin{easylist}[itemize]
	& Network Filter Level
	& SSL Termination
	& Protocol Validation and Rebuilding
	& Character Encoding and Unicode Verficatoin
	& White/Black List Filter
	& Cookie Protection
	& URL Encrypting
	& Smart Form Protection
	& Response Content Filter
	& Response Rewriting
\end{easylist}

\subsection{Reverse Proxy}
\todo[inline]{Reverse Proxy}
\subsection{Pre-Authentication}
Der Zweck einer Pre-Authentication ist dass jegliche \textbf{Backend Requests} bereits \textbf{authentifiziert} sind. Zudem werden \textbf{forensische und Log-Daten} abgelegt für spätere Verwendung.
\subsection{Filtering}
\todo[inline]{Filtering}
\subsection{Unique ID}
\todo[inline]{Unique ID}
\subsection{Smart Filtering und URL Encryption}
\todo[inline]{Smart Filtering und URL Encryption}