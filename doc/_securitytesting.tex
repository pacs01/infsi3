\section{Security Testing}

\subsection{Notwendigkeit}
Die Bedrohung von Informationssystemen ist allgegenwärtig und man kann jederzeit ein Ziel eines Angriffes werden. Auch aus der Sicht der Beteiligten gibt es viele verschiedene Risiken, welche als Treiber für ein Security Testing dienen.
\begin{table}[H]
	\begin{tabularx}{\textwidth}{l|X}
		\textbf{Rollen} & \textbf{Risiken}\\ \hline
		Firmeninhaber, -Teilhaber & Finanzielle Risiken, Verlust von Assets, Reputationsrisiko\\ \hline
		Aufsicht, Regulatoren & Schädigung des Wirtschaftsplatzes, Reputationsrisiko\\ \hline
		Verwaltung, Schulen & Druck der Aufsichtsbehörde\\ \hline
		Entwickler, Ingenieure, Forscher & Know-How Abfluss\\ \hline
		Sicherheitsverantwortliche & Jobverlust\\ \hline
		Kunden, Nutzer & Persönlichkeitsschutz (Datenschutz, Finanzielle Risiken)\\ \hline
	\end{tabularx}
	\caption{Risiko der Beteiligten in verschiedenen Rollen}
\end{table}

Bei einer Sicherheitsprüfung muss die Bedeutung aller Ebenen (Prozess-, Applikations- und Infrastruktur-Ebene) beurteilt werden. Es muss sichergestellt werden, dass der Prüfbereich sinnvoll (in Abhängigkeit der Risikoeinschätzung) festgelegt wird.

\todo[inline]{Ablauf}
\todo[inline]{Begriffe}