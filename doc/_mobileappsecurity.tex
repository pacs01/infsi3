\section{Mobile Application Security}

\subsection{iOS}
Basierend auf OSX hat Apple das iOS lanciert. Dabei wurde Objective-C, Swift, und C verwendet. iOS verfügt über Sandbox Mechanismen, ein Data Protection API, Code Signing und den Appstore mit manueller Freischaltung und App Review durch Apple.

\begin{figure}[H]
	\centering
	\includegraphics[width=0.6\textwidth]{./img/mobileappsecurity_iOS-DataProtectionAPI}
	\caption{iOS Data Protection API}
\end{figure}

\begin{description}
	\item[File Key] unique for each file
	\item[Hardware Key] embedded into Crypto Chip
	\item[Class Key] depends on passcode and device
	\item[File System Key] derived from Hardware Key
\end{description}

Bei iOS ist das \textbf{Permission Model} so eingerichtet, dass bei jeder Verwendung nach Erlaubnis gefragt wird und individuell entschieden werden kann. Das \textit{App-Manifest} beinhaltet keine Permissions.

\subsection{Android}
Auf Linux basierend in Java und C entstand Andoid, welches von verschiedene Hardware-Anbietern übernommen wurde. Somit hat es auch noch viele alte Versionen auf dem Markt.

Android unterstützt SD Karten, Sandboxing via JVMs, Code Signing, und mehrere verschiedene Appstores.

Die \textbf{App Permissions} sind bei Android im Manifest hinterlegt und können ab Android 6 individuell angenommen oder abgelehnt werden.

\subsection{Windows Phone}
Windows Phone von Microsoft unterstützt ebenfalls strict sandboxing, Address Space Layout Randomization (ASLR, zufällige Ladeposition von Programmen in Speicher als Schutz gegen Overflow), Bit Locker Verschlüsselung, Data Protection API (ähnlich wie bei iOS), aber auch Gefahren wie automatisches Wi-Fi credential sharing über Facebook, Outlook oder Skype. (In Version 1607 wurde die umstrittene Funktion \textit{Wi-Fi Sense} wieder entfernt.)

\subsection{Xamarin}
Xamarin ermöglicht die Entwicklung von platformübergreifenden nativen Apps. Dabei wird das Mono .NET Framework verwendet und in C\# programmiert, welches zu IL-Code (Microsoft Intermediate Language) kompiliert wird.

Die native APIs werden zwar zu den .NET namespaces gemapped, jedoch müssen platformabhängige Security Features sowie User Interfaces immer noch für jede Platform geschrieben werden.

\subsection{Cordova (Phonegap)}
Cordova wird verwendet um Mobile Apps mit HTML, JS und CSS zu entwickeln. Es bringt somit native Funktionalität, basiert aber auf der WebView Komponente mit all ihren bekannten Schwachstellen. Alternative: eigene native Bridges schreiben und eine hardened WebView verwenden.

\section{OWASP Mobile Top 10}
\begin{figure}[H]
	\centering
	\includegraphics[width=0.8\textwidth]{./img/OWASP_MobileTop10}
	\caption{OWASP Mobile Top 10, 2016}
\end{figure}

\subsection{M1: Improper Platform Usage}
Falsche Verwendung eines Platform Features oder fehlende Sicherheitsmassnahmen, und somit das Verletzten von Guidelines. \\

\textbf{Beispiel}: Zu viele Permissions, unverschlüsseltes Ablegen von Credentials in Property-File anstelle der KeyChain (iOS). \\

\textbf{Lösung}: Konsultieren der Platform Guidelines beim Entwickeln.

\subsection{M2: Insecure Data Storage}
Durch Ablegen von Daten in unsicherem Speicher oder unvorhergesehenen Datenleaks (siehe folgende Unterkapitel) besteht eine erhöhte Gefahr, wenn das Gerät abhanden kommt oder sich mit Malware infiziert (z.B. nach jailbreak / rooting).

\subsubsection{System Screenshots}
Das Betriebssystem erstellt automatisch einen Screenshot der App, um diese unter den zuletzt verwendeten Anwendungen oder im Task Manager darzustellen.\\

\textbf{Beispiel}: E-Banking Kontoinformationen können eingesehen werden über den Screenshot der App.\\

\textbf{Lösung}: Screenshots vom System unterbinden oder sensitiven Daten überdecken.
\begin{lstlisting}[language=C, caption=Lösung für iOS]
(void) applicationDidEnterBackground:(UIApplication *)application{
	// mask the sensitive data or add an additional layer on top			
}
\end{lstlisting}
\begin{lstlisting}[language=Java, caption=Lösung für Android]
window.addFlags(LayoutParams.FLAG_SECURE, LayoutParams.FLAG_SECURE);
\end{lstlisting}

\subsubsection{Keyboard}
Viele Keyboards verfügen über Autocomplete. Apples iOS zum Beispiel speichert bis zu 500 Wörter.\\

\textbf{Beispiel}: Passwort Feld unterstützt Autocomplete; Das Passwort wird somit auch an anderen Stellen vorgeschlagen, wenn mit den selben Buchstaben begonnen wird. \\

\textbf{Lösung}: Autocomplete bei sensitiven Daten deaktivieren.
\begin{lstlisting}[language=C, caption=Lösung für iOS]
theTextField.secureEntry = YES;
theTextField.autocorrectionType = UITextAutocorrectionTypeNo;
\end{lstlisting}
\begin{lstlisting}[language=Java, caption=Lösung für Android]
android:inputType="textNoSuggestions";
\end{lstlisting}

\subsubsection{Clipboard}
Standardmässig wird ein globales Clipboard für Copy \& Paste verwendet und jede App kann darauf zugreifen.\\
\textbf{Beispiel}: Sensitive Daten befinden sich noch im Clipboard aus der E-Banking App.

\textbf{Lösung}: Clipboard leeren oder ein dediziertes Clipboard verwenden.
\begin{lstlisting}[language=C, caption=Lösung für iOS]
UIPasteBoard *uniqueBoard=[UIPasteboard pasteboardWithUniqueName];
\end{lstlisting}

\subsubsection{Logs}
Crash- und Applikationslogs können sensitive Daten enthalten. Unter iOS werden im Apple System Log (ASL) alle Einträge bis zum nächsten Reboot gecached (nicht in der Sandbox). Debug Logs sollten bei der finalen App immer deaktivieren werden.\\

\textbf{Beispiel}: Bei gescheitertem Login-Versuch werden im Log Benutzername und Passwort gespeichert.\\

\textbf{Lösung}: \textbf{Keine sensitiven Daten loggen} und dedizierten Logger für Debug Build der App verwenden.

\subsubsection{Web Data}
Eine WebView speichert jegliche Daten wie ein konventioneller Web Browser. Dies umfasst einen Web Cache, lokalen Speicher, Cookies und Passwörter, die direkt oder in eine SQLite Datenbank gespeichert werden. \\

\textbf{Beispiel}: Ablegen von unverschlüsselten Passwörtern und Benutzernamen in SQLite Datenbank.\\

\textbf{Lösung}: Korrekten Systemspeicher verwenden.

\subsubsection{Inter-Process Communication}
Apps können URL Handlers registrieren. Andere Apps können dann darauf Daten senden, wobei der Empfänger nicht verifiziert wird. Zudem entsteht ein Problem, wenn mehrere Apps den selben Handler registrieren. Bei Android kann dann die aufzurufende Applikation ausgewählt werden, bei iOS wird automatisch die zuletzt installierte App verwendet.\\

\textbf{Beispiel}: Sensitive Daten werden bei einem Aufruf über einen Handler abgefangen.
\begin{lstlisting}[language=XML, caption=Aufruf von Skype]
skype://090012312312?call
\end{lstlisting}

\textbf{Lösung}: Bei Android direkter Aufruf eines Intent aus der anderen App.

\subsection{M3: Insecure Communication}
In diese Kategorie fallen beispielsweise das Verwenden von inkorrekten oder alten SSL Versionen, schwache Protokollaushandlung oder gar plaintext Kommunikation.

Ab iOS 9 wurden mit \textbf{App Transport Security} minimale Anforderungen eingeführt (TLS 1.2 mit PFS, Zertifikat und SHA-256 Signatur, 2048-bit RSA oder 256-EC Key). Apps entsprechen diesen Vorbestimmungen oder müssen explizite Ausnahmen definieren.

\subsubsection{Certificate Pinning}
Standardmässig wird bestimmten Certificate Authorities (CA) vertraut. Deren Root Zertifikate sind auf jedem Gerät vorinstalliert.\\

\textbf{Beispiel}: Die CA wurde gehacked oder ein Attackierer hat dem Gerät eine falsche CA hinzugefügt. \\

\textbf{Lösung}: Certificate Pinning, eingeführt durch Google Chrome, speichert das Zertifikat oder dessen Public Key in der App und überprüft dieses bei der Verwendung.

\subsection{M4: Insecure Authentication}
Das Verletzten der Privacy durch Verwenden von Geräte-spezifischen Daten, Spoofing, vorhersehbaren Session IDs oder weglassen eines Session Timeouts.

\subsection{M5: Insufficient Cryptography}
Schwache Ciphers und/oder Keys, ungeschützte Keys in der App gespeichert, oder selbstgemachte Krypto-Algorithmen.\\

Android besitzt die \textit{Bouncy Castle Library}. Alternativ kann diese auch als Klon in die App eingebunden werden, welches dann Spongycastle genannt wird. iOS hat das Common Crypto API CCCRypt. Die KeyChain kann zudem für asymmetrische Kryptographie verwendet werden und alternativ steht OpenSSL zur Verfügung.

\subsection{M6: Insecure Authorization}
Zugangsberechtigungen für nicht autorisierte Benutzer oder das Erlauben von Operationen an nicht dafür vorgesehene Benutzer. Ein klassisches Beispiel ist \textit{Forced Browsing}, das Abrufen von nicht referenzierten, aber trotzdem erreichbaren Inhalten.

\subsection{M7: Client Code Quality}
Implementationsprobleme wie Buffer Overflows, XSS, format string Schwachstellen etc.\\
\textbf{Lösung: Sollte direkt im Code gefixed werden.}

\subsection{M8: Code Tampering}
Es besteht die Möglichkeit dass Code selbst verändert oder ausgetauscht wird. Dazu gibt es mehrere Möglichkeiten:

\begin{description}
	\item[Binary / Monkey Patching] Runtime Patch ohne Kopie des originalen Source Codes.
	\item[Resource Modification] Manipulation von lokalen Ressource-Files, welche z.B. URLs beinhalten.
	\item[Method Hooking] Bei erweiterbaren Frameworks das Missbrauchen von Hooks für eigene Zwecke.
	\item[Method Swizzling] Implementation zweier Methoden austauschen.
	\item[Jailbreak / Rooting] Root Access zu System erlangen, kann aber anhand von Filestrukturen, Libraries oder OS-abhängigen Merkmalen erkannt werden.
\end{description}

\subsection{M9: Reverse Engineering}
Der Code kann eingesehen werden durch Binary Inspection, Dissasembly oder geleaktem Source Code. Gegen die ersten zwei Fälle kann ein \textbf{Code Obfuscator} verwendet werden, wie etwa ProGuard im Android SDK. Grundsatz: Verlass dich nicht auf \textit{Security by Obscurity}.

\subsection{M10: Extraneous Functionality}
Schwachstellen wie Backdoors für Entwickler, Funktionen die nicht für den Release gedacht wurden oder Passwörter in Kommentaren.

\subsection{Secure Coding Checkliste}
\begin{easylist}[itemize]
	& Data at Rest
	&& iOS: DataProtection && Keychain API
	&& Backup Exclusion
	&& Caching
	& Data in Motion
	&& Certificate Pinning
	&& Transport Security
	& Data Leakage
	&& Prevent side-channel leakage: logs, keyboard, screenshots...
	& Input Validation
	&& Input sanitization e.g. URL handler parameters
	&& Format strings
	& Code Protections
	&& Code Obfuscation
	&& Stack protection
	&& Anti-Debug controls
	&& Compiler settings
	&& Code Injection Checks
	& Web View Hardening
	&& Disable local file access
	&& Disable plugins
	&& Disable javascript	
	&& URL whitelisting
	& Environment integrity
	&& Jailbreak / Rooting detection
	&& Version control / mandatory updates
\end{easylist}

