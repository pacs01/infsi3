\section{Einleitung}

\subsection{Information Security Management}
Kleiner Rückblick zu den Begriffen aus InfSi1:
\begin{description}
	\item[Asset] Wert aus Sicht der Organisation, ob materiell oder immateriell, Informationen oder Dienstleistungen.
	\item[Threat] Bedrohung, möglicher Grund für einen ungewollten Vorfall, der das System oder die Organisation schädigen kann.
	\item[Vulnerabilities] Schwachstelle einer Schutzmassnahme, die durch eine oder mehrere Bedrohungen ausgenutzt werden kann.
	\item[Controls] Gegenmassnahme als Mittel zur Risikohandhabung.
	\item[Gefährdung] Zusammenspiel von Asset, Threat und Vulnerabilities.
	\item[Applied Threat] Die Gefährdung ist eine Bedrohung, die konkret über eine Schwachstelle auf ein Objekt einwirkt. (Bedrohung und Schaden)
	\item[Risiko] = Wahrscheinlichkeit eines Zwischenfalls * Schaden = Bedrohung * Verletzlichkeit * Schaden
\end{description}

\subsection{Treiber für Informationssicherheit}
Als hauptsächliche Treiber der Informationssicherheit zählt die \textbf{Konformität zu Gesetz und Vorschriften}.

Folgendes liefert ein Teil von Gesetzen und Vorschriften in den verschiedenen Sparten:
\begin{easylist}[itemize]
	& Bearbeiter von Personendaten
	&& Diverse Strafgesetzbuch Artikel
	&& Datenschutzgesetz (DSG)
	&& US Health Insurance Portability and Accountability Act (HIPAA)
	& Finanzdienstleister
	&& Bankgesetze (Bankgeheimnis, etc.)
	&& EU Directive on Payment Services (PSD)
	&& Payment Card Industry Data Security Standard (PCI)
	& Telecom/ICT-Anbieter
	&& Fernmeldegesetz
	&& Lawful Interception
	& Allgemeines Controlling
	&& Sarbanes-Oxley Act (SOX), US-börsenkotierte Firmen
\end{easylist} 

\subsubsection{Datenschutzgesetz DSG}
\begin{quotation}
	Wer als Privatperson Personendaten bearbeitet oder ein Datenkommunikationsnetz zur Verfügung stellt, sorgt für die Vertraulichkeit, die Verfügbarkeit und die Richtigkeit der Daten, um einen angemessenen Datenschutz zu gewährleisten. - DSG Art. 8
\end{quotation}

\subsubsection{Strafgesetzbuch StGB}
\begin{quotation}
	Missbrauch einer Fernmeldeanlage - StGB Art. 179septies (Virentatbestand)
\end{quotation}

\begin{quotation}
	Unbefugtes Beschaffen von Personendaten - StGB Art. 179novies (Hackingtatbestand)
\end{quotation}

\subsubsection{Health Insurance Portability and Accountability Act HIPAA}
Umfasst mehrere Regeln zu Datenschutz und -Sicherheit. Dazu werden auch eine Regel für die \textit{Unique Identifiers} definiert (National Provider Identifier, NPI).

\subsubsection{Bankengesetz BankG}
\begin{quotation}
	\ldots ein Geheimnis offenbart \ldots [oder] zu einer solchen Verletzung des Berufsgeheimnisses zu verleiten sucht. - BankG Art. 47
\end{quotation}

\subsubsection{Payment Card Industry Data Security Standard}
Dieser umfasst mehrere Bestimmungen wie z.B. der Einsatz einer \textbf{Firewall}, \textbf{Verschlüsselte Übertragung}, \textbf{Beschränkung des physikalischen Zugriffs}, usw.

\subsubsection{Sarbanes-Oxley Act SOX}
Fordert verschärfte interne Kontrollsysteme und führt zu höheren Anforderungen an die \textit{Corporate Governance}.

\todo[inline]{Bedrohungen}