\section{OWASP Top 10}
Die OWASP Top 10 sind die zehn wohl wichtigsten Schwachstellen von Anwendungen, welche aus dem \textit{Open Web Application Security Project} hervor ging.\\

Die aktuellste Version ist 2013 in \fnurl{http://owasptop10.googlecode.com/files/OWASP\%20Top\%2010\%20-\%202013.pdf}{Englisch} erschienen und in mehrere Sprachen übersetzt, unter anderem auch \fnurl{https://www.owasp.org/images/4/42/OWASP_Top_10_2013_DE_Version_1_0.pdf}{Deutsch}. Die hier aufgeführten Kurzbeschreibungen stammen ebenfalls aus der deutschen Version.

\subsection{Injection}
Injection-Schwachstellen (SQL-, OS- oder LDAP-Injection) treten auf wenn nicht vertrauenswürdige Daten als Teil eines Kommandos oder einer Abfrage von einem Interpreter verarbeitet werden. Ein Angreifer kann Eingabedaten dann so manipulieren, dass er nicht vorgesehene Kommandos ausführen oder unautorisiert auf Daten zugreifen kann.

\subsection{Broken Authentication and Session management}
Anwendungsfunktionen, die die Authentifizierung und das Session-Management umsetzen, werden oft nicht korrekt implementiert. Dies erlaubt es Angreifern Passwörter oder Session-Token zu kompromittieren oder die Schwachstellen so auszunutzen, dass sie die Identität anderer Benutzer annehmen können.

\subsection{Cross-Site-Scripting - XSS}
XSS-Schwachstellen treten auf, wenn eine Anwendung nicht vertrauenswürdige Daten entgegennimmt und ohne entsprechende Validierung oder Umkodierung an einen Webbrowser sendet. XSS erlaubt es einem Angreifer Scriptcode im Browser eines Opfers auszuführen und somit Benutzersitzungen zu übernehmen, Seiteninhalte zu verändern oder den Benutzer auf bösartige Seiten umzuleiten.

\subsection{Insecure Direct Object References}
Unsichere direkte Objektreferenzen treten auf, wenn Entwickler Referenzen zu internen Implementierungsobjekten, wie Dateien, Ordner oder Datenbankschlüssel von aussen zugänglich machen. Ohne Zugriffskontrolle oder anderen Schutz können Angreifer diese Referenzen manipulieren um unautorisiert Zugriff auf Daten zu erlangen.

\subsection{Security Misconfiguration}
Sicherheit erfordert die Festlegung und Umsetzung einer sicheren Konfiguration für Anwendungen, Frameworks, Applikations-, Web- und Datenbankserver sowie deren Plattformen. Sicherheitseinstellungen müssen definiert, umgesetzt und gewartet werden, die Voreinstellungen sind oft unsicher. Des Weiteren umfasst dies auch die regelmässige Aktualisierung aller Software.

\subsection{Sensitive Data Exposure}
Viele Anwendungen schützen sensible Daten, wie Kreditkartendaten oder Zugangsinformationen nicht ausreichend. Angreifer können solche nicht angemessen geschützten Daten auslesen oder modifizieren und mit ihnen weitere Straftaten, wie beispielsweise Kreditkartenbetrug, oder Identitätsdiebstahl begehen. Vertrauliche Daten benötigen zusätzlichen Schutz, wie z.B. Verschlüsselung während der Speicherung oder Übertragung sowie besondere Vorkehrungen beim Datenaustausch mit dem Browser.

\subsection{Missing Function Level Access Control}
Die meisten betroffenen Anwendungen realisieren Zugriffsberechtigungen nur durch das Anzeigen oder Ausblenden von Funktionen in der Benutzeroberfläche. Allerdings muss auch beim direkten Zugriff auf eine geschützte Funktion eine Prüfung der Zugriffsberechtigung auf dem Server stattfinden, ansonsten können Angreifer durch gezieltes Manipulieren von Anfragen ohne Autorisierung trotzdem auf diese zugreifen.

\subsection{Cross-Site Request Forgery - CSRF}
Ein CSRF-Angriff bringt den Browser eines angemeldeten Benutzers dazu, einen manipulierten HTTP-Request an die verwundbare Anwendung zu senden. Session Cookies und andere Authentifizierungsinformationen werden dabei automatisch vom Browser mitgesendet. Dies erlaubt es dem Angreifer Aktionen innerhalb der betroffen Anwendungen im Namen und Kontext des angegriffen Benutzers auszuführen.

\subsection{Using Components with Known Vulnerabilities}
Komponenten wie z.B. Bibliotheken, Frameworks oder andere Softwaremodule werden meistens mit vollen Berechtigungen ausgeführt. Wenn eine verwundbare Komponente ausgenutzt wird, kann ein solcher Angriff zu schwerwiegendem Datenverlust oder bis zu einer Serverübernahme führen. Applikationen, die Komponenten mit bekannten Schwachstellen einsetzen, können Schutzmassnahmen unterlaufen und so zahlreiche Angriffe und Auswirkungen ermöglichen.

\subsection{Unvalidated Redirects and Forwards}
Viele Anwendungen leiten Benutzer auf andere Seiten oder Anwendungen um oder weiter. Dabei werden für die Bestimmung des Ziels oft nicht vertrauenswürdige Daten verwendet. Ohne eine entsprechende Prüfung können Angreifer ihre Opfer auf Phishing-Seiten oder Seiten mit Schadcode um- oder weiterleiten.